\documentclass[a4paper,10pt]{article}
\usepackage[utf8]{inputenc}
\usepackage{booktabs}
\usepackage{amsthm}
\usepackage{amsmath}
\usepackage[english]{babel}
\usepackage{mathtools}
\usepackage{amssymb}
\usepackage{amsfonts}
\usepackage{float}
\usepackage{graphicx}

\newtheorem{theorem}{Theorem}[section]
\newtheorem{lemma}[theorem]{Lemma}

\theoremstyle{plain}
\newtheorem{thm}{Theorem}[section] % reset theorem numbering for each section

\theoremstyle{definition}
\newtheorem{defn}[thm]{Definition} % definition numbers are dependent on theorem numbers
\newtheorem{lemm}[thm]{Lemma} % same for lemma numbers
\newtheorem{coro}[thm]{Corolary} % same for corolary numbers
\newtheorem{conj}[thm]{Conjecture} % same for conjecture numbers
\newtheorem{clemm}[thm]{Conjectured Lemma} % same for conjectured lemma numbers
\newtheorem{cthm}[thm]{Conjectured Theorem} % same for conjectured theorem numbers
\newtheorem{ccor}[thm]{Conjectured Corolary} % same for lemma numbers
\newtheorem{obs}[thm]{Observation} % same for observation numbers
\newtheorem{exmp}[thm]{Example} % same for example numbers

\newtheoremstyle{named}{}{}{\itshape}{}{\bfseries}{.}{.5em}{\thmnote{#3's }#1}
\theoremstyle{named}
\newtheorem*{namedtheorem}{Theorem}

% \renewcommand\qedsymbol{$\blacksquare$}
\renewcommand\qedsymbol{QED}

\begin{document}

%opening
\title{FFM Cryptosystem - Finite Field Matrix Cryptosystem - a public-key cryptosystem with encryption and signature schemes based on the hardness of finite field polynomial factorization}
\author{Yuri S Villas Boas}

\maketitle

\begin{abstract}
We introduce a cryptosystem based on the difficulty of factorizing polynomials over finite fields. Public key is given by a full-rank Matrix over this field, and private key by its eigenvector decomposition. Cryptosystem allows for encryption and signature schemes, which are covered in the document, as well as preliminary cryptanalysis.\\

\textbf{keywords:} finite field polynomial factorization, linear algebra over finite field, characteristic polynomial, eigenvector decomposition, public-key cryptosystem, encryption scheme, signature scheme, cryptanalysis.
\end{abstract}

\section{Algebra}

Our proposed cryptosystem can be defined by the table below:
\begin{flushleft}
 \begin{tabular}{|r|c|l|}
    \hline
    \textbf{Component} & \textbf{Formula} & \textbf{Definition}\\
    \hline
     \textbf{Private Key} & $(R, A, R^{-1})$ & $R = \left(\mathbf{r_1}, \cdots, \mathbf{r_n}\right)$ full-rank, with $n \geq 4$, even\\
     &  & $A = \left(a_1\mathbf{e_1}, \cdots, a_n\mathbf{e_n} \right)$, with $B\mathbf{r_i} = a_i\mathbf{r_i} \neq \mathbf{0}$\\
     \hline
     \textbf{Public Key} & $B$ & $B = RAR^{-1}$\\
     \hline
        \textbf{Plaintext} & $P$ & $P\in \mathbb{F}_p^{n\times n}$ represent an ordered basis over $\mathbb{F}_p^n$\\
     \hline
    \textbf{Ciphertext} & $C$ & $C = PBP^{-1}$\\
     \hline
    \textbf{Decription} & $VR^{-1}$ & $V = \left(\mathbf{v_1}, \cdots, \mathbf{v_n}\right)$, where $C\mathbf{v_i} = a_i\mathbf{v_i}$\\
     \hline
    \textbf{Document} & $d$ & $d \in \mathbb{F}_p$\\
     \hline
    \textbf{Signature} & $s$ & $s = \Pi_{i = 1}^{n/2} (x-a_{\pi(i)}^d) \in \mathbb{F}_p[x]$, where $\pi(\cdot) \in S_n$\\
     \hline
    \textbf{Verification} & $s | b_d$ & $b_d = det\left(B^d-xI\right) \in \mathbb{F}_p[x]$\\
    \hline
    \multicolumn{3}{c}{}
\label{tab:notation}
\end{tabular}
\end{flushleft}
where,

\begin{flushleft}
\begin{tabular}{c l}
    $\mathbb{F}_p$ & refers to the finite field of order $p$;\\
    $\mathbb{F}_p[x]$ & refers to the set of polynomials over $\mathbb{F}_p$;\\
    $\left(\mathbf{m_1},\cdots,\mathbf{m_n}\right)$ & signifies a matrix having $\mathbf{m_i}$ as its $i$th column vector;\\
    $\mathbf{e_i}$ & refers to the $i$th canonical vector of $\mathbb{F}_p$;\\
    $p|q$ & means `$p$ divides $q$';\\
    $S_n$ & is the symmetric group of $n$ elements;\\
\end{tabular}
\end{flushleft}

\section{Computability}

Throughout this session we will analyse the \textbf{polynomial-reducibility} of objects defined in the previous session. 
% Here, public information ($n$, $p$, $B$, $C$, $s$, $d$) is, in fact, known, that is, available to the algorithms unless they are explicit being demonstrated to easily yield any of them.
We are particularly concerned about the asymptotic complexity of factorizing elements of $\mathbb{F}_p[x]$ as function of the number of bits of $p$.

Throughout this session, we will have
\begin{equation}
n \geq 2
\end{equation}
\begin{equation}
f(x) = \Pi_{i = 1}^{n}(x-a_i) = c_0x^0+\cdots+c_{n-1}x^{n-1}+x^n : a_1, \cdots, a_n \in \mathbb{F}_p
\end{equation}
\begin{equation}
f, g, h \in \mathbb{F}_p[x] : f = g*h \land deg(g), deg(h) \geq 1
\end{equation}
In other words, $f$ is an $n$-degree monic polynomial over a finite field with $n\geq2$, having $c_i$ as the coefficient to $i$th power, and roots $a_1, \cdots, a_n$ in the field; and $g, h$ non-trivial factors of $f$.

\begin{defn}
We define\footnote{`$p$' in $\leq_p$ stands for \textit{polynomial time}, while `$P$' in the plaintext object stands, in fact, for \textit{plaintext}, and `$p$' in $\mathbb{F}_p$ is an incognate \textit{prime} number. Those are three totally different concepts and they sharing the same letter is just a notation coincidence. Another two things not to be confused are the `polynomials' of the asymptotic complexity of algorithms and the `polynomials' that some of the described algorithms operate on.} the \textbf{non-strict preorder} $a \leq_p b$ as: \textit{``b is polynomially reducible to a.''} namely \textit{``There exists a polynomial-time complexity algorithm to compute $b$ from $a$.''} That definition ensues the following \textbf{strict preorder} $a <_p b$ given by $a \leq_p b \land a \ngeq_p b$, and the \textbf{equivalence relation} $a \equiv_p b$ given by $a \leq_p b \land a \geq_p b$.
\end{defn}

The bases of \textbf{asymmetric} cryptosystems are $<_p$ relations (which are \textbf{asymmetric}). We will now enunciate a few of them component-wise ($\leq_p$ and $\ngeq_p$). The first is the general objects in which the cryptosystem is based, and the following are the objects of the cryptosystem, and how they exactly or approximately replicate the $<_p$ relation. Here (as typically happens), $\ngeq_p$ relations are ultimately based on a conjecture --- \textbf{in}existence of polynomial time algorithms with certain characteristics --- while $\leq_p$ are constructively proven by the definition of the cryptosystem's algorithms.

\begin{lemm}[Easiness of Polynomial Multiplication]
\label{conj:polymulteasy}
\[(g,h) \leq_p g*h\]
\end{lemm}

\begin{clemm}[Hardness of Polynomial Factorization]
\label{conj:polyfacthard}
\[(g,h) \ngeq_p g*h\]
% \[\Pi_{i = 1}^{n}(x-a_i) \nleq_p (a_1, \cdots, a_n)\]
\end{clemm}

\begin{lemm}[eigen value-eigen vector equivalence]
\label{lemm:eveveq}
$(B,R) \equiv_p (B,A)$
\end{lemm}
\begin{proof}
$\,$%linear systems:
\begin{description}
 \item [$(\leq_p)$:] Compute $a_i \coloneqq (\mathbf{e_{j(i)}}^TB\mathbf{r_i})*(\mathbf{e_{j(i)}}^T\mathbf{r_i})^{-1}$, for each $0 \leq i \leq n$ and any $j(\cdot)$ for which $\mathbf{e_{j(i)}}^T\mathbf{r_i} \neq 0$ --- which is guaranteed to exist by $R$ being full-rank. Namely, use the definition of eigenvectors to calculate each eigenvalues by taking one non-zero entry of $B\mathbf{r_i}$ and finding the `ratio' of it to the same (non-null) entry of $\mathbf{r_i}$ (meaning, multiply the former by the later's multiplicative inverse in $\mathbb{F}_p$).
 \item [$(\geq_p)$:] For each $1 \leq i \leq n$, solve $(B-a_iI)\mathbf{x_i}=\mathbf{0}$ on $\mathbf{x_i}$ to find the eigenspace of $\mathbf{r_i}$. Arbitrate any member of that space as $\mathbf{r_i}$. The resulting $R$ and $R^{-1}$ will satisfy $RAR^{-1}=B$.
 
%  Cancel out the degree of freedom to get $\mathbf{r_i}$ by matching any non-null entry of $R\mathbf{e_i}$ and the correspondent (non-null) entry of $\mathbf{x_i}$.
\end{description}
\end{proof}
In other words, either component of the private key, $A$ and $R$ suffice to easily calculate the other ($R$ and $R^{-1}$ obviously are, in the same sense, equivalent too). That allows us to, henceforward refer to the private key as just $A$, for short.

\begin{lemm}
\label{lemm:polmateq}
$f \equiv_p M$, where $M$ is any matrix having $a_1, \cdots, a_n \in \mathbb{F}_p$ as eigenvalues. In other words:

\textit{Factorizing a monic polynomial of degree $n$ known to have all roots in the field, is as hard as finding the eigenvalues of a matrix of order $n$, known to have full eigenvalue decomposition.}
\end{lemm}
\begin{proof}
We have to prove that there are $(\leq_p)$ an easy computation of a matrix $M$ as a function of a given polynomial $f$ with all roots in the field, so that $M$ has full eigenvalue decomposition with the roots of $f$ as eigenvalues; and conversely $(\geq_p)$ an easy computation of a polynomial $f$ as function of a given $M$ known to have full eigenvalue decomposition that makes $f$ have $M$'s eigenvalues as roots.
\begin{description}
\item [$(\leq_p)$:] Take \[M \coloneqq C(f) = \begin{bmatrix}
0 & 1 & 0 & \ldots & 0 \\
0 & 0 & 1 & \ldots & 0 \\
\vdots & \vdots & \vdots & \ddots & \vdots \\
0 & 0 & 0 & \ldots & 1 \\
-c_0 & -c_1 & -c_2 & \ldots & -c_{n-1}
\end{bmatrix} \]
the so called \textbf{companion matrix} of $f$.
\item [$(\geq_p)$:] Compute  \[f(x) \coloneqq \det(xI-M) \]
through \textbf{Gaussian elimination}.
\end{description}
\end{proof}

Now, we will apply the same discussion to each component of the cryptosystem:

\begin{thm}[Completeness of Key Pair]
 \[(R,A(,R^{-1}))\leq_p B\]
\end{thm}
\begin{proof}
Matrix (inversion and) multiplications $B \coloneqq RAR^{-1}$ totalling $\mathcal{O}(n^3)$ scalar multiplications and sums in $\mathbb{F}_p$.
\end{proof}

\begin{cthm}[Soundness of Key Pair]
 \[(R,A(,R^{-1}))\ngeq_p B\]
\end{cthm}
\begin{proof}
From \ref{lemm:eveveq}, ascertaining either component of the private key $A$ or $R$ is sufficient to ascertain the other. From \ref{lemm:polmateq}, this is as hard as factorizing $B$'s characteristic polynomial, which, by \ref{conj:polyfacthard}, is conjectured to be, in fact, hard.%$R$ and $R^{-1}$ are obviously equivalent.
\end{proof}

\begin{thm}[Completeness of Signature]
  \[(A,d) \leq_p s\]
  that is, \textbf{issuing} a signature is easy, and 
  \[(B,s',d') \leq_p (s'|B^{d'}), \forall s', d' \in \mathbb{F}_p[x]\]
   that is, \textbf{verifying} that a signature, corresponds to a documents and a public key is easy.
\end{thm}
\begin{proof}
$\,$
\begin{description}
\item [issuing:] $s$ is given by \[s \coloneqq \Pi_{i = 1}^{n/2} (x-a_{\pi(i)}^d)\] which has polynomial comple as a direct application of lemma \ref{conj:polymulteasy}.%, for some arbitrary $\pi(\cdot) \in S_n$ (that is productory of $(x-a_i^d)$ for $n/2$ arbitrary $a_i$'s). Clearly of polynomial time complexity.
\item [verification:] Computation of $B^{d'}$, its characteristic polynomial $b_{d'}$ and division of it by $s$, yielding: \[\mathtt{s\_is\_valid} \coloneqq ((det(B^{d'}-xI)\%s) == 0)\] where $\%$ represents the remainer operation $==$ is an equality test. All components are of polynomial time complexity.
\end{description}
\end{proof}

\begin{cthm}[Soundness of Signature]
  \[s \ngeq_p (B,d)\]
  Meaning, a signature cannot be easily forged, that is, computed without $A$.
\end{cthm}
\begin{proof}
In order to forge a valid signature $s$ for document $d$ and public key $B$, an adversary would have to partially factorize $b^d = det(B^{d}-xI)$, which is conjectured to be hard in \ref{conj:polyfacthard}. In this case, adversary has a polynomial $det(B-xI)$ whose roots are known to be $d$th roots of those of $det(B^{d}-xI)$. This additional knowledge is not known to ensue lower time-complex cryptanaly, but is suspected to allow for statistical attacks.
\end{proof}

\begin{thm}[Completeness of Encryption]
  \[(B,P) \leq_p C\]
  that is, \textbf{encrypting} a plaintext message is easy, and
  \[(A,C) \leq_p P\]
  that is, \textbf{decrypting} a ciphertext message is easy.
\end{thm}
\begin{proof}
$\,$
\begin{description}
\item [encrypting:] Matrix inversion and multiplications $C \coloneqq PBP^{-1}$ totalling $\mathcal{O}(n^3)$ scalar multiplications and sums in $\mathbb{F}_p$.
\item [decrypting:] We use the algorithm described in $(\geq_p)$ session of lemma \ref{lemm:eveveq}, and the fact that $C$ is known to have a full eigenvalue decomposition given by $C = VAV^{-1}$ with (privately) known $A$ to ascertain a valid $V$, then proceed to compute a $P \coloneqq VR^{-1}$. By construction, the obtained $P$ will satisfy $C = PBP^{-1}$ and therefore represents the same basis meant by the sender of the plaintext. All components are of polynomial time complexity.
\end{description}
\end{proof}

\begin{conj}[Soundness of Encryption]
  \[P \ngeq_p (B,C)\]
  that is, deciphering a ciphertext is hard.
\end{conj}
% \begin{proof}
\textit{Argument.} The apparent hardness of ascertaining $P$ from $C = PBP^{-1}$ seems to be analogous to that of eigenvalue decomposition, proven to be equivalent to the conjectured (\ref{conj:polyfacthard}) hardness of polynomial factorization.% Namely, it is to be proven the \textbf{in}existence...
% \end{proof}

% 
% \begin{conj}
% \label{conj:prentry01p}
% Probability of fortuitous ocurrence of any entry whatsoever equal $0$ is polynomially bounded by $1/p$.
% \end{conj}
% 
% Intuitively, it seems to be the case that assuming the random elicitation of the private key is, in fact, uniform, and that the hashing algorithm of documents is ideal, and that the specification of vectors of the eigenspace for the plaintext is uniformly random, and that the plaintext itself is uniforrmly random\footnote{That is actually very plausible seeing that public key encryption schemes typically are used to encrypt symmetric session keys represented by arrays of random bits.}, the probabilty of any given entry of the \textbf{public key}, a \textbf{signature} and a \textbf{ciphertext} being $0$ is bounded to a power of $1/p$ per instance of usage. Anecdotal observations suggest that such fortuitous occurrences $0$'s for realistic values of $p$ are, in fact, negligible.
% 
% \begin{obs}
% One relevant implication of conjecture \ref{conj:prentry01p} being true, the treatment such possibility is just a matter of formal mathematical correctness, and implementations of this algorithms could be made to simply ignore them, as merely contemplating them could cost more than blindly taking the risk run-time failure, however catastrophic the consequences.
% \end{obs}
% 
% \begin{obs}
% Another relevant implication conjecture \ref{conj:prentry01p} being true is that it to be the case that the difficulty hazardly yielding a $0$ entry as described above could be used as a cryptographic scheme of some sort.
% \end{obs}
% 
% \begin{obs}
% hard to formalize, but `It's unlikely that two ratios between non-zero entrie would coincide by happenschance (presumed proability of 1/(p-1), so the cryptanalytic process of assessing if a candidate to eigenvector is in fact eigenvector, should verify the coincidence of two such ratios, therefore economizing the verification of other entries.
% \end{obs}
% 
% \begin{obs}
%  
% \end{obs}

\section{Complexity}

\section{Cryptanalysis}

\section{Dedication}

This work is dedicated to Dr. Hans-Christian Herbig, a noble, honorable man who kept unswerving virile dignity in face of calumny, persecution, malatie and death. His standing by truth, righteousness and to the personal defense of myself, then, a complete stranger, when nobody else would, at peril to his very career, and without any prospective gain whatsoever will never be forgotten.

\end{document}
